\section{Review of Literature}
Optimization of water distribution system is a complex problem and involves numerous interrelated decision. Goncalves and Pato \cite{Goncalves2000} divided the problem into three subproblems that they used for their three-phase procedure. Many other studies deal only on some of these subproblems. In summary, these subproblems include decisions concerning:
\begin{itemize}
    \item the links of the network / layout (type A decisions)
    \item the flow passing through the links (type B decisions)
    \item the specifications of the hydraulic components (type C decisions)
\end{itemize}
In their three-phase procedure, it begins with determining the layout of the pipes using a heuristic algorithm. From the layout, the flows of fluid on each link are calculated using a formula. Finally, the specifications of the hydraulic components are determined using mixed binary linear programming.

Because of the complexity of the problem when all decisions are considered, the problem has been tackled by most studies by solving the subproblems separately. Many of the studies focused only on the type C decisions and assumed the layout to be predetermined \cite{Goncalves2000,Lee1984,Lejano2006}. A famous paper by Alperovits and Shamir \cite{Alperovits1977} solves pipe diameters, pump capacities and reservoir elevations, and operational parameters (type C decision) using linear programming gradient. Contrarily, Lee \cite{Lee1984} considered type A and B simultaneously. It was concerned with a branched pipe network system that transports fluid or gas from multiple sources to multiple demand nodes. The junction locations are determined simultaneously with the selection of pipe sizes and pump capacities to minimize the capital cost and operating cost using a nonlinear programming model. Another study that considered type A and B simultaneously is made by Lejano \cite{Lejano2006}. It optimizes the layout and design of branched pipeline water distribution systems using a mixed integer linear programming. But instead of simply minimizing the cost, it also maximizes the net benefits for the demand nodes.

Many of the recent studies used evolutionary solution finding algorithms such as simulated annealing, ant colony optimization, shuffled frog leaping algorithm, and genetic algorithm. Simulated annealing was used by Cunha and Sousa \cite{Cunha1999}, and Goldman and Mays \cite{Goldman2005}. Eusuff and Lansey \cite{Eusuff2003} develop a shuffled frog leaping algorithm for the problem. Ant colony optimization was used by Maier et al. \cite{Maier2003}. Vasan and Simonovic \cite{Vasan2010} used differential evolution which is an improved version of genetic algorithm.

Currently, there already exist studies that used Steiner tree problems to model the optimizations on pipelines. Goncalves and Pato \cite{Goncalves2000} used an algorithm for Steiner forests to determine the layout of their pipes. Prodon et al. \cite{Prodon2010} solved leakage detection on water distribution network by solving prize-collecting Steiner problem. Li et al. \cite{Li2015} highlighted the importance of Steiner trees in their survey of optimization methods for oil-gas pipeline network layouts.
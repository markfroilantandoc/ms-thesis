\section{Methodology}
\subsection{Problem Instance}
This research will find optimal layouts for branched pipe networks with multiple sources. There will be sources nodes and demand nodes that should be connected. Optional junction nodes could be placed anywhere in the graph to help improve the network. The pressure and flow requirements of the demand nodes are known and flow capacities of the sources are also given. The diameters of each pipe link are selected from a finite set of available diameters with corresponding unit prices. The best solution is the layout with the least cost computed using the formula below.

\begin{equation}
\label{cost_function}
C_T = {\sum CD_kL_{ijk}} + {\sum CP_s}
\end{equation}

\begin{align*}
    \text{where:}& \\
    C_T &- \text{total cost} \\
    CD_k &- \text{unit cost of pipe with diameter $k$} \\
    L_{ijk} &- \text{length of pipe from $i$ to $j$ with diameter $k$} \\
    CP_i &- \text{pumping cost at source $s$} \\
\end{align*}

\subsection{Flow and Head Loss Computations}
The flows for each link can be easily computed using flow conservation law where the amount of inflows in a node should be equal to the outflows. To calculate head loss due to friction in the pipe links, the widely used Hazen-Williams equation is adopted.

\begin{equation}
\label{hazen_williams}
S = \frac{10.67Q^{1.85}}{C^{1.85}d^{4.87}}
\end{equation}

\begin{align*}
    \text{where:}& \\
    S &- \text{head loss per unit length of pipe} \\
    Q &- \text{volumetric flow rate in $m^3/s$} \\
    C &- \text{pipe roughness coefficient} \\
    d &- \text{pipe diameter in $m$} \\
\end{align*}

\subsection{Approaches}
Some studies generate layout designs while simultaneously considering pipe sizes in order to find the best solution. However, the others determine the pipe sizes after finding a layout in order to reduce the computation time and reduce the complexity of the problem. This research aims to compare the solutions generated by the two approaches.

\subsubsection{Separate}
The separate approach will first find a layout of the nodes and links with junction nodes. The layout will be determined by finding the optimal Steiner points from the graph created by the given source and demand nodes where GeoSteiner \citep{Juhl2014}, a software package for computing Steiner trees in a plane, will be used. From the given layout, simulated annealing (SA) will be used to determine the best pipe sizes for the links.

\subsubsection{Simultaneous}
For the simultaneous approach, SA will be used for both the location of junction nodes and the pipe sizes. $j$ nodes will be randomized which will serve as initial locations of junction nodes where $j$ is $\text{number of source and demand nodes} - 2$. $j$ is the maximum possible number of Steiner points given a graph as cited in \cite{Lee1984}. The minimum spanning tree from the given source and demand nodes will serve as the initial state of the layout with random pipe sizes. A neighbor state will be a change in one of the pipe sizes of the links or a change in the location of the junction nodes.

\subsection{Simulated Annealing}
\begin{center}
\begin{figure}[h]
    \begin{algorithmic}
        \State $solution \gets \text{random solution}$
        \State $temp \gets \text{initial temperature}$
        \State $cost \gets \text{C(solution)}$
        \While{$temp > 0$}
            \State $i \gets 1$
            \While{$i < N$}
                \State $solution^{\prime} \gets neighbor(solution)$
                \State $cost^{\prime} \gets C(solution^{\prime})$
                \If{$cost^{\prime} < cost$}
                    \State $solution \gets solution^{\prime}$
                    \State $cost \gets \text{C(solution)}$
                \Else
                    \If{$e^{\sfrac{\Delta d}{temp}} > random(0,1)$}
                        \State $solution \gets solution^{\prime}$
                        \State $cost \gets \text{C(solution)}$
                    \EndIf
                \EndIf
                \State $i \gets i+1$
            \EndWhile
            \State $temp \gets temp \times coolingFactor$
        \EndWhile
    \end{algorithmic}
    \caption{Simulated Annealing}
    \label{al_simulated_annealing}
\end{figure}
\end{center}

Simulated annealing (SA) is a probabilistic method described by \cite{Kirkpatrick1983} and \cite{Cerny1985} for finding global optimum of a given function. It emulates the process of repeated heating and controlled cooling of solids to make it more workable. The algorithm is outlined in Figure \ref{al_simulated_annealing} as described by \cite{Cerny1985}.